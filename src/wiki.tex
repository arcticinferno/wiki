\documentclass {report}

% PKGS
\usepackage{sectsty}
\usepackage{hyperref}
\usepackage{listings}


% SETTINGS

\allsectionsfont{\centering}
\setcounter{tocdepth}{3}
\setcounter{secnumdepth}{3}

% CMDS

% #1 Name #2 Last Name #3 Site name #4 Date #5 Link
\newcommand{\csite}[5]{
  #2, #1. \textit{#3}. #4. \textit{\url{#5}} \newline
}
\newcommand{\link}[4]{
  #1. \textit{#2}. \textit{\url{#4}}. #3 \newline
}


% About
\title{Wiki}
\author{arcticinferno}


\begin{document}

\maketitle

\tableofcontents


\chapter{Introduction}

\section{What is this wiki aiming to do?}
Inside this wiki/document, we intend to document various parts of Linux that you may encur while switching. I have compiled a list of things that I intend to cover, and will put them in a section just ahead. This document will consist of everything we write, but I intend to eventually "port" it to a website format, but for now, it should be fine as a document.

\section{What we will document}

I will include a tree of things that will be documented ahead, but I would first like to thank the people from the following reddit threads for giving me a bunch of things to add.


\link{Reddit}{r/linux}{April 2023}{https://www.reddit.com/r/linux/comments/12lyruw/im\_going\_to\_write\_a\_full\_progression\_guide\_for/}

\link{Reddit}{r/linux}{April 2023}{https://www.reddit.com/r/linuxquestions/comments/12lzryp/im\_going\_to\_write\_a\_full\_progression\_guide\_for/}


\textbf{Alright, now that we've got that out of the way, here are the things I'm going to document:}
\textit{*also please note that this is not in order}

\begin{enumerate}
  \item Hardware
    \begin{enumerate}
      \item To Be Expanded (Supported hardwarem, good brands, etc.)
    \end{enumerate}
  \item Use Cases
    \begin{enumerate}
      \item Developing Software
      \item Gaming
      \item System Administration
      \item To Be Expanded
    \end{enumerate}
  \item Installation
    \begin{enumerate}
      \item What is a Linux Distribution?
        \begin{enumerate}
          \item What makes up a "distro"
            \begin{enumerate}
              \item The Kernel
              \item The Shell
              \item The Applications
              \item The Package Manager(s)
            \end{enumerate}
          \item Should I use a small distro? Should I use a large distro?
          \item Why you shouldn't use Kali Linux as a desktop.
        \end{enumerate}
      \item Burning A USB Drive
      \item Booting into the BIOS
      \item Dual Booting - Benefits \& Disadvantages
      \item Installing Linux
    \end{enumerate}
  \item System
    \begin{enumerate}
      \item Mounting Drives
      \item Automounting Drives
      \item File Structure
      \item Networking
      \item System Configuration
      \item System Administration
      \item Users
    \end{enumerate}
  \item Useful Resources
    \begin{enumerate}
      \item Arch Linux Wiki - https://wiki.archlinux.org/
      \item Gentoo Linux Wiki - https://wiki.gentoo.org/
    \end{enumerate}
  \item Gaming
  \item Virtual Machines
    \begin{enumerate}
      \item KVM
      \item Xen
      \item Virtual Box
      \item QEMU + \{KVM, Xen\}
    \end{enumerate}
  \item Advanced Parts Of A Linux System
    \begin{enumerate}
      \item Types Of Distributions - Source Based / Binary Based
      \item Init Systems
      \item Boot Loaders
      \item Kernel(s)
      \item Filesystems
    \end{enumerate}
  \item Trying out more advanced software
    \begin{enumerate}
      \item Window Managers
      \item Text Editors
      \item Terminal Emulators
    \end{enumerate}
\end{enumerate}

\chapter{Getting Started}
\section{Installation}
The majority of these steps will work for any Linux distribution, but I will specifically document the process for pop_os and artixlinux.
\subsection{Prerequisites}
You will need the following thing(s) to install Linux:
\begin{enumerate}
  \item A USB drive with >= 4GB of space
\end{enumerate}

\subsection{Pop! OS}
To get started, we will download the official ISO from their website. (https://pop.system76.com/) Also, to note, if you have an NVIDIA GPU, you should select the nvidia option whilst downloading. This will make it a lot easier in the future.

\subsection{MacOS}
In MacOS you won't need to download any additional software to burn the ISO to your USB drive. Simply navigate to the folder in which you downloaded the ISO, and run the following commands:

\begin{lstlisting}
lsblk
\end{lstlisting}

This will show you the drives that you can install to.

\begin{lstlisting}
sudo dd if=<iso> of=/dev/<drive> status=progress bs=4M
\end{lstlisting}

After the command fully finishes, simply remove the drive.

\end{document}
